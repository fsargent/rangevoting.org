To CRV members...

A possible way to generate interest in range voting, to get more CRV members, and to get
more endorsements of RV, is to write Letters to Editors of various magazines.
I here am posting all the recent letters I have written so far.  

The problem is, such letters are rarely actually published.  BUT if you ALSO write
"independent" letters to the same magazine at about the same time, then it seems likely 
they WILL publish one of them.  I am making that easy for you to do by placing my 
letters here, along with the email addresses of who to send it to.

I think my letters tend to be too long.  Editors like short.
Make sure all letters give the URL    http://RangeVoting.org

--wds

If you write your own letters then please let me have a e-copy for me to post here.
Also feel free to suggest/do other magazines beyond these.
(Ms. Magazine?   The Progressive?   Christian magazines?  The Economist?)

SOME E-ADDRESSES FOR MAJOR PAPERS:  (but minor papers are far more likely to accept)
letters@latimes.com
letters@globe.com
ctc-TribLetter@Tribune.com
http://www.csmonitor.com/cgi-bin/contactus.pl
letters@time.com
letters@newsday.com
editor@nysun.com
wsj.ltrs@wsj.com
letters@washpost.com
letters@nytimes.com
letters@sfchronicle.com
opinion@seattletimes.com
letters@mercurynews.com
http://www.nypost.com/postopinion/letters/letters_editor.htm
voicers@edit.nydailynews.com

-------------------------------------------------------------------------
Unfortunately I no longer have my letter to the editor of THE NATION
   http://www.thenation.com/
The Editor   The Nation 
33 Irving Place, 8th Floor 
New York, NY 10003
   http://www.thenation.com/contact/lett   <-- form to submit e-letters to editor
But it was important because it was in response to a piece by Ralph Nader in
THE NATION a couple weeks ago.  If this letter is published then Nader will
pay some attention to it.  We want to get Ralph Nader's endorsement of Range Voting.
-------------------------------------------------------------------------

LETTER TO THE EDITOR Harper's Magazine   letters@harpers.org

Thank you, Mark Crispin Miller, for your superb story
on the Ohio 2004 election shenanigans and the USA media's "see no evil" 
approach to not covering it.

But unfortunately _Miller also_ exhibits a "see no evil" approach
too-typical of the shallow media in his remark
"The infamously factious Democrats were fiercely unified -
Ralph Nader garnered only about 0.38 percent of the national vote."
What is wrong with this remark?  It misses the elephant in the room, which is
that the Democrats are intentionally turning on their apparent natural _ally_, Nader,
by, e.g, filing over 20 simultaneous harassment lawsuits against him, and in particular 
shutting Nader off the Ohio ballot so he could not get any votes whatever in Ohio.
This self-destructive behavior only makes any sense because of the USA's
inherently flawed "plurality voting system."  In a better voting system in
which the phenomenon of "vote splitting" did not exist, the Democrats instead would
have welcomed Nader's candidacy.

Such a better voting system exists: "_range voting_"!
In it, each voter gives a score from 0 to 99 to each candidate, and the candidate with
the highest average score wins.  (Example: a valid range vote might have been 
"Nader=99, Kerry=99, Bush=0, Cobb=47.")  In this system, voting 99 for Nader in no
way diminishes Kerry's votes.  So you can concentrate on voting honestly instead of
on voting and acting dishonestly by, e.g, pretending Nader is the enemy.  Really,
our undemocratic voting system is the enemy.  Voters thusly saying much more
information in their vote, and saying it more honestly, implies far better democracy.

To learn more about range voting and its many non-obvious enjoyable properties, 
and our activism campaign to get it adopted in the Iowa 2008 caucuses,
please examine the web site of the Center For Range Voting (CRV)
   http://math.temple.edu/~wds/crv .

Warren D. Smith,  PhD mathematician and founder of the CRV.

------------------------------------------------------------

To REASON MAGAZINE
attn: 
jsanchez@reason.com bdoherty@reason.com


This is a "letter to the editor".  Right now it is near-minimal length, but if you are
interested I can expand it into a somewhat longer piece.  But I think this is definitely 
worth bringing to the attention of libertarians.  These are _italics_.
------------------------

"RANGE VOTING" IS THE ONLY WAY FOR LIBERTARIANS TO SURVIVE
To Libertarians everywhere:

Neither the Libertarian, nor any other "third party,"  is never going to achieve
any substantial electoral success in the USA for as long as the USA employs
the current "plurality voting system."
This voting system causes it to be strategically stupid for anybody to vote
third party (it actually _hurts_ their cause) and studies show that about 90%
of US voters who honestly rank a third party candidate highest, in fact for 
strategic reasons vote major-party.  The more important your third-party ideas are, 
the _more_ justified this strategic-betrayal decision becomes.  This huge 90% 
loss-factor means you will never succeed.  This self-reinforcing 2-party dominance
is so well known and supported by so much evidence that it is called "Duverger's law."

Admit this instead of lying to yourselves about it.

This means the top survival-priority mission of Libertarians and third-parties everywhere
must be: improve the voting system.    I am a mathematician and the founder
of an activist group pushing for _range voting_.  In this system, each voter
gives a score from 0 to 99 to each candidate 
(e.g.: Badnarik=99, Bush=37, Kerry=0, Nader=74)
and the candidate with the highest total wins. 
Badnarik and Cobb foolishly advocated Instant Runoff Voting in their debates,
but unfortunately IRV (for non-obvious reasons) also leads to Duverger
law 2-party domination - so that was suicidally stupid of them both.
Other non-obvious facts:
 * Range voting can be done on all the USA's current voting machines without modification.
 * Range voting is immensely more beneficial to small third parties than other systems.
 * Range voting should be wanted, and more than any other system, by all third
    parties and all major parties also, for use in Iowa 2008 caucuses.

For this and more information about range voting, including the world's only study 
of how US voters range-vote for Libertarians, and the world's only computer
simulation study comparing different election systems including range, and the fact that
Libertarian Vice-Presidential candidate Richard Campagna has endorsed us,
please browse our activist website
   http://math.temple.edu/~wds/crv .
You can then join or endorse our organization, and/or the range voting bulletin 
board, including our effort to make range voting happen in Iowa 2008.

Warren D. Smith,   Mathematician and founder of Center for Range Voting 

------------------------------------------------------------------------------
(To ed. please INCLUDE the URL given at the end of this letter.)

LETTER TO THE EDITOR   International Socialist Review  editors@isreview.org

Lance Selfa ["Right Wing Republic? Bush's Victory..." ISR 39, Jan/Feb 2005]
analysed why Bush won in great detail and complexity.  However, analyses of this sort 
ignore the elephant in the room, which is: the USA is hardly a "democracy" at all.
Only 2 parties dominate the US picture - the total number of members of
a third party occupying US House or Senate seats, or high executive
or court positions (about 600 positions in all) is currently exactly zero.
As far as that is concerned, there is no worse democracy on the planet.
This means voters have at most 2 choices.  But in fact they usually have only
one choice.  That is because the re-election rate for US congressmen (and
seats in most state houses too) is 98%.  That is higher than in any other democracy.   
US congressmen are more likely to die in office than lose an election.
Polishing up Kerry's "image" about "moral values" isn't going to affect this.
It is almost irrelevant.  What you should be asking is, why is there such massive
2-party domination, why is everything so rigged, and what can we do about it?

There is, however, a genuine reform available.  It is called "range voting."
In this simple voting system, the voter rates each candidate on an 0-99 scale
and the one with the highest average score wins.  There is every reason to believe
that just this simple step will cure the above-mentioned ills of US democracy,
indeed yielding a better government than in democracies with more primitive voting 
systems.  The whole reason for 2-party domination of voter fear of  "vote splitting"
causing the least-worst major party candidate to lose.  Vote splitting no longer exists
with range voting.  Computer simulations suggest that the total societal benefit from
adopting range voting would be comparable to the benefit from inventing democracy in
the first place.  Think of it as a tool for society to make better decisions - exactly
what a Socialist journal like yours should be concerned about.  To learn more
about range voting (& other systems), why range voting can be done easily with the voting
machines we already have, a feasible strategy for making the USA adopt it, etc,
please examine our web site and then join our effort.

--Warren D. Smith, mathematician and founder of the 
Center For Range Voting,   http://math.temple.edu/~wds/crv  ;
Mike Ossipoff, longtime voting-reform activist.

-------------------------------------------------------------------------

LETTER TO THE EDITOR, MOTHER JONES MAGAZINE   backtalk@motherjones.com

Clinton Hendler ("NYC, meet IRV", 21 Sept. 2005)
argues that New York City should switch to IRV
("Instant Runoff Voting," the single-winner special case of
the STV, "Single Transferable Vote," proportional representation
multiwinner election system), notes the "biggest success story to date"
for IRV which was that it got adopted by San Francisco.
but worries that IRV may be "just too complicated for voters to understand."

Unfortunately, Hendler neglected to tell the reader some
inconvenient but important facts:
1. New York City already _did_ adopt STV in 1936. 
However, a referendum in 1947 forced its abandonment and a return
to the plurality voting system.
2. In San Francisco's IRV election, every contested race was stopped dead
by flawed election software, everything had to be redone, and the announcement of
the election results had to be delayed for weeks - not exactly a "big success."
[San Francisco Chronicle, 4 Nov 2004, front page.]
3. IRV cannot be run on the "lever" voting machines used throughout New York State.  
Completely new voting machines designed for IRV would have to be got.
4. IRV makes near-tie nightmare scenarios like in Florida 2000 _more_ likely,
and can exhibit some very disturbing properties, such as the fact that your vote ranking
Gore top, can actually _cause_ the defeat of Gore, whereas if only you'd voted Bush top,
then Gore would've won.

Changing the voting system is indeed a good idea, but IRV is not the best answer.
A better system is "range voting."
In it, each voter gives a score from 0 to 99 to each candidate, and the candidate with
the highest average score wins.  (Example: a valid range vote might have been
"Nader=99, Kerry=99, Bush=0, Cobb=47.")  

Range voting is simpler than IRV.   
It can be run on every voting machine in the USA, including New York's lever machines,
right now, without modification.  
If your range-vote ranks some candidate top, that can never hurt them.
Range voting makes ties less, not more, likely.  
Range voting has been adopted by many organizations (e.g. the Olympics uses it to
vote for gold medal gymnasts) and to my knowledge,
no organization that adopted range voting has ever decided to abandon it.
In range voting, voters express opinions on an 0-99 scale about _every_ candidate,
as opposed to only being allowed to express an opinion about _one_ candidate.
And if you vote 99 for Nader, that in no way diminishes Kerry's votes. 
So you can concentrate on voting honestly instead of
on voting and acting dishonestly by, e.g, Democrats pretending Nader is the enemy.  
Really, our undemocratic voting system is the enemy.  In range voting, the phenomenon
of "vote splitting" simply does not exist, so there is no reason for Democrats to
fight against their natural ally Nader.   Range voters thus say much more
information in their vote, and say it more honestly.  Result: far better democracy.

To learn more about range voting and its many non-obvious enjoyable properties,
and our activism campaign to get it adopted in the Iowa 2008 caucuses,
please examine the _web site_ of the Center For Range Voting (CRV)
   http://math.temple.edu/~wds/crv .

Warren D. Smith,  PhD mathematician and founder of the CRV.
------------------------------------------------------------------------
To the Editor
American Conservative
letters@amconmag.com

Conservatives like Pat Buchanan who question the 2-party duopoly
need to know about an important idea for letting them participate
in the game instead of merely yelling from the sidelines
(or perhaps rushing in rather like a "streaker").

That idea is a better, more democratic voting system called "range voting" (RV).
This system eliminates the phenomenon of "vote splitting" so "must
vote for the lesser of two evils" thinking disappears.  In RV voters
rate each candidate on an 0-99 scale or "no opinion."  Our web site
   http://math.temple.edu/~wds/crv
analyses the idea in detail and many non-obvious facts are shown such as
that RV can be done quite simply on today's voting machines right now (try the "demo"!);
that "instant runoff voting"  _still_ leads to 2-party domination and hence
is "fool's gold" for third-party candidates like Buchanan; and that 
RV is wholy Constitutional.

We believe the Democrat and Republican Parties will _want_ RV for their Iowa 2008
caucuses since each will gain the advantage by adopting it.  Help us achieve that.

Warren D. Smith
PhD Mathematician and CRV founder.
----------------------------------------------------------
July 30, 2004

mail polemics@chroniclesmagazine.org
letter to editor about "range voting"

To the editor:
Samuel Francis (30 July 2004) asks
"Should you waste a vote on Michael Peroutka?"
noting that
"If you'd like to cast a ballot for a conservative this year, forget George W. Bush."

He misses, however, the elephant in the room, which is:
why should "voting honestly" _be_ the same as "wasting your vote"?  Why cannot you
_both_ vote honestly, _and_ have an equally powerful vote as people who vote dishonestly?

A superior voting system, "range voting" (RV) achieves that goal, allowing the USA to 
actually have a democracy in which voters can actually express their opinions,
for a change.  In RV voters rate each candidate on an 0-99 scale or 
"no opinion."  Our web site
   http://math.temple.edu/~wds/crv
analyses the idea in detail and many non-obvious facts are shown such as
that RV can be done quite simply on today's voting machines right now (try the "demo"!);
that "instant runoff voting"  _still_ leads to 2-party domination and hence
is "fool's gold" for third-party candidates like Peroutka; and that
RV is wholy Constitutional.

We believe the Democrat and Republican Parties will _want_ RV for their Iowa 2008
caucuses since each will gain the advantage by adopting it.  Help us achieve that.

Warren D. Smith
PhD Mathematician and CRV founder.
-------------------------------------
letter to editor about "range voting"

To the editor, TNR  letters@tnr.com:
Letter to the Editor
The New Republic
1331 H Street, NW Suite 700 
Washington, DC 20005
(202) 508-4444

"Welcome to the Hackocracy"
fails to look beyond skin deep, failing to ask _why_ the hacks control the asylum
and how to change the system to prevent it.  The reason is current Republican control of every
branch of government, leading nominees whose sole qualification is party loyalty,
to pass easily.  (Republican votes for Bush nominees have been over 1000:1 in favor.)
Now go deeper.  The reason for 1-party control is we have 2-party control with
98%  congressional re-election rates, i.e. 1-party control.  ("Democracy"?  I think
not.)  The reason for 2-party control (of the top 600 people in the government, _not one_ 
is a member of a "third party") is the _voting system_:  if you want to vote for
a third-party, then you are being _foolish_ with the present system and actually _hurting_ their
cause.  Does this matter?  Do drowned cities matter?

A superior voting system, "range voting" (RV) fixes the problem at the core,
allowing the USA to actually have a democracy in which voters can actually express 
their opinions without being foolish, for a change.  
In RV voters rate each candidate on an 0-99 scale or 
"no opinion."  Our web site
   http://math.temple.edu/~wds/crv
analyses the idea in detail and many non-obvious facts are shown such as
that RV can be done quite simply on today's voting machines right now (try the "demo"!);
that "instant runoff voting"  _still_ leads to 2-party domination and hence
is "fool's gold" for third-party candidates like Nader; and that
RV is wholly Constitutional.

We believe the Democrat and Republican Parties will _want_ RV for their Iowa 2008
caucuses since each will gain the advantage by adopting it.  Help us achieve that 
key trial.

Warren D. Smith
PhD Mathematician and CRV founder.
-------------------------------------
To the editor
New York Times Magazine
magazine@nytimes.com

Concerning redistricting:
My organization, the "Center for Range Voting,"
has put forward a simple proposal for drawing districts in
a completely unbiased automated manner via a simple algorithm 
consisting of repeatedly splitting the state 
by means of the shortest possible eligible split-line.  
At our web site, http://math.temple.edu/~wds/crv (click "gerrymandering" center bottom)
you can find maps of various states' current districts as well as the far simpler
unbiased ones that our algorithm would produce at far less cost.

Warren D. Smith
Mathematician and CRV founder.
http://math.temple.edu/~wds/crv
warren.wds@gmail.com
wds@math.temple.edu
631-675-6128
------------------------------------------------------
In the below I use _this_ to denote italics.
The CRV website subpages I refer to here are
http://math.temple.edu/~wds/crv/TarrIrv.html
http://math.temple.edu/~wds/crv/rangeVirv.html
http://math.temple.edu/~wds/crv/Irvtalk.html
===============================================
Letter to the editor, THE NATION

I am a mathematician and the founder of the
Center For Range Voting (CRV)
http://math.temple.edu/~wds/crv .

Greg Dennis's letter advocating Instant Runoff
Voting (IRV) in the 7 Nov. NATION, unfortunately
contains falsehoods:

* "Under IRV... votes are counted so as to insure the winner always has a
majority of support."

False. The CRV's web site's IRV discussion
contains an example election in which an
IRV-loser candidate C would beat each other candidate head-to-head
(including the
IRV winner D) - based on the very same preference-order-votes used in the
IRV election - by approximately a 2:1 margin.

* IRV "sidesteps the Nader effect... allows
third-party candidates to run without causing
the major-party candidate ideologically closest
to them to lose."

False. The CRV's web site's IRV discussion
contains two example IRV elections where Dennis's
"1. Nader, 2.Kerry, 3. Bush" voter causes both
Nader and Kerry to lose, whereas if that
voter had voted "1. Kerry" and ranked Nader & Bush either 2,3 or 3,2, then
Kerry would have won. This shows how IRV voters who like Nader
best can still be motivated to "betray" him
in their vote, which is presumably exactly
why every IRV country in the world
is 2-party dominated. Unfortunately the
USA's Green Party has (idiotically) endorsed
IRV and seems unaware of this key fact.

Both of these counterexample elections are realistic.
One of the CRV's missions is to educate
misinformed groups such as the Green Party
and FairVote.org about the true properties
of voting systems. A system superior
in quality and simplicity to IRV, and
which genuinely _does_ permit Nader voters
to rank him top without _ever_ thereby
causing Kerry to lose, is _range voting_:
Also, the range voting winner always beats
every other candidate head-to-head in a
2-outcome range-election based on those
same range votes. RANGE VOTING:

a. Voters give every candidate a score from a
fixed numerical range such as 0-to-9.
(Example vote: Nader=9, Kerry=9, Bush=0, Cobb=5.)
b. Candidate with greatest average score wins.

Range voting is familiar from the Olympics.
Non-obvious facts also shown on the CRV
web site:
* Range Voting can be run on every voting
machine in the USA, right now, with no modification needed. (Unlike IRV.
Dennis, while mentioning
San Francisco adopted IRV, neglected to mention
that their special new IRV voting machines failed dramatically in their
first use: every nonobvious election triggered a disastrous software failure
and the announcement of the results was delayed weeks.)
* Range voting has been endorsed by both the Libertarian and Socialist 2004
Presidential
candidates.
* Our 2004 exit poll study showed that Range voting would have given Nader,
and every other third party candidate, over 50 times more votes relative to
Bush.

I repeat: Every. Over 50 times.

Warren D. Smith
(PhD in applied math from Princeton 1998).
---------------------------------------------------------------

 gerrymandering - simple proposal to eliminate
NOTE: in the below letter we write _italics_.  Please be sure not to remove
the CRV web site's URL from the letter, since it gives far more details than
I possibly can in a couple of paragraphs.

-----------------
To the editor
Washington Post
letters@washpost.com

Eli Rosenbaum in his 29 Oct op-ed piece about CA and OH redistricting
claimed that independent agencies in practice were no solution to
the gerrymandering problem, citing WA, AZ, and NJ states as examples.
While we agree those states (especially AZ) do appear gerrymandered,
we remark that the independent district-drawer approach has worked superbly 
in _Canada_.

A simple solution which genuinely always _does_ work
has been put forward by my organization, the "Center for Range Voting."
It is to draw all districts in a completely unbiased automated manner via a simple 
algorithm  consisting of repeatedly splitting the state 
by means of the shortest possible eligible split-line.  
At our web site, http://math.temple.edu/~wds/crv (click "gerrymandering" center bottom)
you can find maps of various states' current districts as well as the far simpler
unbiased ones that our algorithm would produce at far less cost.  
No carefully selected bi- or non-partisan commission is needed.

Warren D. Smith
PhD Mathematician and CRV founder.
http://math.temple.edu/~wds/crv
warren.wds@gmail.com
wds@math.temple.edu
631-675-6128
