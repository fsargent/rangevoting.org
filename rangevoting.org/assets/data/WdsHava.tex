MY COMMENTS on the proposed EAC "voluntary Voting System Guidelines."
-------------------Warren D. Smith---Sept 2005-----------------------

Before I begin, let me introduce myself: I am Warren D. Smith.  I am a
mathematician writing a book (nearly complete) about voting.  I have a PhD in
applied math from Princeton.  I am the inventor of the currently
theoretically-best-available cryptographically-secure voting protocol,
recently presented by me at the FEE 2005 conference in Milan.

GENERALLY:
These guidelines are 2 book volumes long and essentially unreadable.
Checking whether a voting machine obeys the guidelines is an extremely
hard problem - in fact beyond the resources of all of humanity and certainly
beyond the resources of typical US counties - and yet no agency or
entity is created by these guidelines with the technical wherewithal to have 
any chance of testing machines to see if they meet the guidelines.  That
means the guidelines are a joke.

But it gets worse. It seems to me that the guidelines largely consist of 
identifying bad things about the status quo, and then fossilizing them
as "approved by these guidelines".


SECTION 6.7: WIRELESS:
Imagine a law intended to protect us from nuclear weapons that read "it
is perfectly ok for you to own a nuclear weapon and keep in in your house, although we
look down upon that practice.  All we ask (as a purely voluntary guideline,
not required by law) is this: if you do choose to own a nuclear weapon,
then you must sign a piece of paper saying you agree only to use it in the
following authorized ways."

Well that is precisely what this ludicrous section does.  It essentially
says to possibly-corrupt voting machine manufacturers 
  (and let me note: the major US voting machine 
  companies contain known convicted criminals - bribery, theft, fraud,
  drug trafficking -
  in high management positions, and quite possibly employees who have
  been bribed 
  by foreign intelligence agencies - in fact it is wholy legal for those 
  employees or even owners to literally BE foreign nationals who ADMIT
  to also being employees of foreign intelligence agencies... and
  indeed US voting machine companies HAVE been owned by
  rich foreigners from non-democratic countries, who almost certainly were 
  closely linked to that country's rulers): 
Hey!  It is perfectly ok if you put wireless communications
devices in your voting machines which would allow anybody within a mile
to reprogram or control those machines to do whatever he wants - PROVIDED you sign a
piece of paper (e.g. see 6.7.2.5 & 6.7.5.2  for typical laughably
unenforceable examples - actually checking
to see if these criteria are satisfied is well known to be a
Turing-undecidable problem) saying "I certify my wireless interfaces will not 
be used in those bad ways."

There is a lot of mumbo jumbo about cryptography.  That is a red
herring. What matters is: at any point in time a corruptor could emit the right
radio message which would cause it to transition into a new "do whatever you
want" mode that officially was claimed not to exist.  This mode could have been
inserted secretly by a programmer.  No feasible amount of testing would be able
to discover the existence of this secret mode and the fact it did not
always obey the claimed crypto protocols, because the magic "open sesame" message could be 300
random-looking bits long and depending in a secret way upon the time so that any
failure of this sort would be essentially irreproducible (since by 2.2.5.2.1b the
voting machine must include a real-time clock).  No amount of manual inspection of the
computer program inside the voting machine (even if it were available for
inspection) would necessarily find this trapdoor because it is well known to be a
Turing-undecidable problem to determine whether a computer program
contains such a trapdoor.  See Marvin Minsky: "Finite & Infinite Machines" for info about
Turing-undecidability. Possible result: the end of democracy in the USA.

Is the convenience of wireless really worth the very plausible risk of
the end of our entire democracy?  Hello!  I don't think so!

Ban all free-space electromagnetic communication methods into every
voting machine.
Period.

ELECTRIC POWER:
The current guidelines permit information to flow into the voting
machine via signals injected into electric power lines.  No precaution 
whatever is required.
(This technology is commercially available in your local Radio Shack.)
That is an outrage.  Ban all use of power lines for any sort of
communication
into or out of the voting machine.

SPOILED BALLOTS:
Suppose the voter feeds in an invalid ballot (such as an "overvote").
Then I want it to be a requirement that the voting machine IMMEDIATELY
notify the voter his ballot is invalid & rejected - and permit that
voter to try again.
It should be absolutely unacceptable for a voting machine to simply
silently accept an invalid ballot without notifying the voter there is any
problem - and indeed it should not be possible to adjust the machine to make it be so 
silent - but these guidelines simply permit this practice without even a disparaging
comment!

This is the year 2005.  Unintended spoiled ballots simply should not
exist.

NATIONWIDE BALLOT STANDARDS:
It is an outrage that the USA does not have uniform nationwide standards for ballot format,
placing us well behind most other democracies and allowing huge scope for election manipulation,
as well as just wasting time and causing confusion.

PAPER TRAILS:
First of all, these guidelines do not mandate a "voter-verified paper trail" i.e. a
record of all the votes cast, readable by an unaided human.  (I do not think there is
any reason to demand that this record actually literally be on "paper" -
"Babylonian clay tablets" would be even better as far as I am
concerned.)

That non-mandate is an outrage.
I believe that all voting machines containing an electronic computer
should be required to be accompanied by such a paper trail.  There have
already been instances of voting machine failure leading to uncorrectable errors due
to the lack of such trails - and this includes cases where that data loss prevented
determining the winner. There have already been instances where voting machines have
been used and later proven on videotape to vote for candidates other than the
one  the voter selected, with no way to correct the error in the election
result.

However, fortunately, there are numerous guidelines in sec 6.8
concerning what voting machines that miraculously DO happen to go beyond guideline
requirements by providing paper trails, should be like, and I largely agree with them.

But 6.8.7.5 is a silly completely undefined requirement.  Here would be
an improvement:
"the (paper) record should withstand 5 hours exposure to 150C and -50C
temperatures, exposure to direct Florida sunlight in an air atmosphere for 1 year, 
or soaking in water for 1 week followed by careful drying.  It should
withstand a 2 Tesla orientation-changing magnetic field for 1 minute."

SUPPORT FOR IMPROVED VOTING SYSTEMS:
The "plurality" voting system predominantly used in the USA at present
(in which a "vote" is the "name of a single candidate") is well known
to be severely flawed, and this has been known since the 1780s.
For that reason, many people would like to see it replaced by a superior
system.  Other systems have been and are being used in various locations in
the USA and abroad, and are wholy constitutional.
Predominant among the superior systems is "range voting" where your
vote is a "numerical score from 0 to 99 awarded to each candidate" for example
a range vote might be "Kerry=99, Bush=0, Nader=99, Badnarik=37".  
The email bulletin board
   http://groups.yahoo.com/group/RangeVoting/
is devoted to range voting and click on the "related link" near the bottom of
that page to be brought to an educational and advocacy site for range voting.

I would like to see the guidelines recommend that voting machines provide support
for improved voting systems.  It is known that range voting elections can be run on
any voting machine that handles multiple plurality elections (without 
modification of the machine).  This is good since it means fairly painless transitions
between plurality and range voting are possible.  But still, this scheme engenders a
certain amount (depending on the machine type) of inconvenience
for either the election administrators or the voters, and it would be better if
the machine itself supported this type of voting directly and by design.

LIMITS ON COMPUTING POWER AND SECRET MANIPULABILITY:
If a voting machine contains a computer, then it should be demanded
that the computer contain 
   * at most 1 Kbyte of random-access read/write memory, including
   "registers" and "cache".
   * unlimited amount of read-only random access memory (ROM) but all
   ROM is required to be built-in and not easily replaceable.
   * unlimited amount of write-only random access memory.
   * unlimited amount of unidirectional-sequential-access read/write
   memory.
(In the above, "memory" includes both semiconductor, magnetic, or any
other kind, and any removable rewriteable media such as disks.  "Write-once"
memory such as PROMs or writeable optical disks does NOT count as "read-only" -
it counts are "writeable" even if the voting machine itself does not have that
writing capability.)
   * The clock rate of the computer shall be at most 10 MegaHertz.
   * All software must be made public by the manufacturer,
    on a US government web site, at least a year before the machine is
    used and throughout the operation of the machine, and in
    both compiled and source-code forms, and with that source code
    agreed by experts to be "clearly written".
   * As soon as anybody spots a bug and reports it to that government
   agency, all bug-fixes to the software must be paid for by the voting machine 
   manufacturer and note that every such bug fix and paying is to include
   the 1-year public pre-viewing requirement for the new modified program
   and hence the payment must include the cost of the 1-year required
   downtime for the buggy machine.
    * Anybody who does spot and report such a bug must be paid a reward
by the voting machine manufacturer, where the first reward is the price
of that voting machine, and each successive reward is double the value of the
previous one until the the price reaches the USA per capita income or 16 times
the price of that machine, whichever is higher.
This is to continue until such time as the manufacturer declares these
machines to be incapable of meeting requirements and hence must all be
withdrawn from use - and if that happens the manufacturer must refund half of
all purchase prices of those machines.

These here rules are intended to make it impossible or at least
difficult to "reprogram" the computer without wholesale replacement of the ROM.
That in turn allows us to at least hope to obtain reasonable confidence the machine is 
actually running the program it was advertised to be running.  Also the difficulty and
expense of replacing the ROM should tend to incentivize manufacturers to provide
only simple programs without bugs.  
It is technically possible to make computer programs bug-free - although
only comparatively small, simple, and clear ones.
But under the current guidelines there is little or no motivation for
manufacturers to try to do so, as far as I know the manufacturers have never adopted
bug-free software techniques, and the whole cost and responsibility of
debugging the software is shifted to the US taxpayer and to some undefined and
unprovided testing and certifying agency.  With the rule changes I recommend here, the
voting machine manufacturer would actually be required or at least highly incentivized
to do what it takes to provide bug-free code.

You may ask: why are my rules demanding that the computer be a primitive, low-capacity
sort of computer?  The answer is that voting only requires a low-capacity computer.
So this is no handicap!  It is, however, a handicap when it comes to installing enormous
bug-filled untested unclear programs.  Which is good because we want to prevent that
and force all programs, and I mean "all," to be small and clear.
(Realize: you can make the evil code actually eat itself
so that it cannot be detected afterwards.  You can make evil code invisible in the source code
by means of language-redefintion techniques.  This is known to have been done.  But these
cheating techniques are not easy to do if the code is tiny.)

In particular it is simply outrageous to allow voting machines to run
enormous programs whose code is secret and undoubtably partially written by paid foreign
secret agents, and containing huge bug counts, such as "Microsoft Windows".  But the
present guidelines, outrageously, permit that.  (Why not simply tell the
USSR to run our elections for us?  Wouldn't that be simpler?)

A computer program 100 million lines long containing just a few lines of
rogue code could do extremely dangerous things inside a voting machine, and no
checking procedure whatever by anyone is capable of guaranteeing spotting that
rogue code, because that checking problem is well known to be Turing-undecidable.  
Voting machines in use in the USA are already known to contain trapdoor code and
continue to be used anyway.   Thus computers are an extremely dangerous thing to
have inside a voting machine.  For that reason, if they are there, we want to 
maximize the chances the program is valid.
If the program is small and on view by everybody in the world for 1 year, 
that maximizes chances of spotting such trapdoors.  If the program is kept secret, made
technically easy to change, and is allowed to be huge and to consist of "obscured"
code - all of which is outrageously permitted by the proposed guidelines - then
that minimizes the chances.

Again my strategy with this rule is to make a rule that is checkable
and that is not Turing-undecidable, and that removes the nuclear weapon
rather than handing them out on streetcorners but with the caveat that the 
recipients must sign pieces of paper saying they will not misuse them.
(Here "large alterable computer programs" are playing the role of the
"nuclear weapon".)

CRYPTOGRAPHIC PROTOCOLS:
One of the hopes for the future of voting is to employ "cryptographic protocols" which 
enable certain mathematical guarantees about vote privacy and election correctness to be made.
These schemes are based on "zero knowledge proof & verification protocols."
They offer the potential for an immense increase in election validity and fraud prevention
far above that ever previously achieved.  The guidelines leave the entire subject unaddressed.
I suggest formulating at least a definition of what such a crypto-secure voting system IS,
and then offering to allow more powerful computers in voting machines satisfying that definition.

The definition should involve votes being 
  *private: a voter who wishes to keep his vote anonymous and secret should be able to do so
  (with mathematical certainty under cryptographic assumptions).   Note: this means
  secret even from the machine which itself receives that vote (it receives it only
  in encrypted form).

  *valid: zero-knowledge proofs must be produced that the vote is valid

  *zero-knowledge proofs must be output by the voting system that only pre-registered voters voted and
   no double votes were used

  *the correct election result must be produced, with zero-knowledge proof of correctness

---
Warren D. Smith
