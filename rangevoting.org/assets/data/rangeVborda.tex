WHY RANGE VOTING IS BETTER THAN BORDA COUNT VOTING
--------------Warren D. Smith----Nov 2004---------


1.WHAT THEY ARE.
Range voting: In an N-candidate election, each vote
is an N-tuple of numbers each in the range 0 to 99.
The Kth number in the tuple is a "score" for candidate K.
You take the average of all the Kth entries to find
the average score for candidate K.  The candidate
with the highest average score wins.
(Voters are allowed to leave an entry BLANK to denote
"don't know anything about that candidate."  Blank entries
not incorporated into average.)

Borda: Each vote is an ordering of the N candidates
from best to worst.  (Voters are NOT allowed to omit a
candidate they know nothing about, and are NOT allowed
to regard two candidates as EQUAL.)
The top ranked candidate gets a score of N-1, the bottom ranked 0,
and more generally the Kth-ranked gets a score of N-K.
All scores are summed and the candidate with the highest
total score wins.

Range and Borda voting are actually very similar as far as their advantages are concerned,
but differ a lot with respect to Borda's disadvantages.
(One way to see it: the only difference between 
Borda and RV is that RV lets you <i>choose</i> what point assignments you'll give 
to the candidates, making RV more democratic, expressive,  and responsive 
than Borda; while Borda sometimes makes the choice for you.)


2.STRATEGIC BORDA VOTING.
The biggest problem with Borda is that it reacts very badly to "strategic voting."

I remember one time when I worked for NEC Research Institute and we had to vote to decide
who, among about a dozen candidates, to hire.  There were several camps, each favoring a 
different candidate who excelled in one way or another.  There were also many mediocre 
candidates - nonentities - whom nobody particularly wanted.  Arguments grew impassioned.

So then our boss, who definitely was several watts shy of being the brightest 
bulb in the box, said "we have to be fair about this. Let's vote."  
And he right then (apparently) independently reinvented the Borda count
system and told us all to provide our Borda votes (preference orderings) - by
secret ballot of course, since he'd read in some managerial
handbook that secret ballots were better.  Then he'd add them up.

Well, of course, since everybody there was an arrogant pushy scientist, everybody
quickly figured out that the thing to do was to rank your favorite first,
then artificially rank all of his perceived major rivals artificially last.
Finally, by the rules of the Borda system, the non-entities had to be ranked
somewhere in the middle.  (Of course, you *could* honestly rank all the top candidates
top, but that would be like having one-tenth of a strategic vote, unacceptably weak.
Nobody could afford to be that stupid and ineffectual.  Plus, there was no public
embarrassment about submitting lies in your ballot, since everybody's ballot was secret.)
Then, my boss made a great show of closing his eyes and shuffling all the ballots... 
then opened them and added them up... and golly... that was odd... the 
most-favored candidates all seemed to be ones that had been dismissed as
nonentities before... they must have a good deal more support than he'd 
realized...  hmm... the result-order really seems quite illogical and random... 
but after all, this clearly IS the most fair possible voting system, so we have to
concede that Mr. Putz really is the truly most-favored candidate, unexpected though
he may be... ok, meeting over, thanks for all your input, folks.

NEC Research Institute eventually collapsed and nearly all its scientists were fired.
This particular meeting, in its small way, was one contribution to its downfall.
(The parent company NEC still survives, thanks to a bailout by the Japanese government.)

   AS A PICTURE:  if the 3 good candidates are A,B,C then the strategic votes are:
    A > nonentities > B > C      (cast by about 1/3 of the voters)
    B > nonentities > C > A      (cast by about 1/3 of the voters)
    C > nonentities > A > B      (cast by about 1/3 of the voters)
   ----------------------------------------------------------------
    total:  A,B, and C each get an average score of about N/3, 
       whereas the nonentities get, on average, a score of about N/2.
       So a nonentity always wins and the 3 good candidates always
       are ranked below average.  In this kind of scenario Borda actually
       performs worse than plurality voting.

Incidentally, this particular pathology is both very serious (near-pessimal candidate
elected) and very common (occurs whenever 3 there are comparable rivals) and it 
happens not only under Borda voting, but under *every* voting system of "Condorcet type."
This severe reaction to strategic voting seems a very good reason to discard
all Condorcet-type voting methods!


3.CONTRAST THIS WITH STRATEGIC RANGE VOTING.
Strategic voting also hurts us in the range system, but not nearly as much.  For example,
in the scenario above, nobody would feel strategically forced to middle-rank the nonentities, 
and in fact, they would, quite honestly, rank them last.  Hence, all the votes would
consist of 99s and 0s for the good candidates, and 0s for the nonentities.
(The 0s for the good candidates would be dishonest strategic attempts to hurt
rivals.)  Despite this strategic dishonesty, one of the good candidates
would still necessarily win.


4. VOTER EXPRESSIVITY - and WHAT IF THE VOTERS REALLY ARE HONEST?
Well,  they can be more honest and more expressive with range voting than with 
Borda voting, so we would expect better results.

In range voting, voters can express the idea that they think 2 candidates are equal.
In Borda, they cannot. 

Range voters can express the idea they are IGNORANT about a candidate and want to leave the
task of rating him to other, hopefully more knowledgeable voters.  In Borda, they
can't choose to do that.

Borda voters who decide, in a 3-candidate election, to rank A top and B bottom,
then have no choice about C - they have to middle-rank him and can in no way express
their opinion of C.   In range voting, they can.

If you think  A>B>C>D>E,  undoubtably some of your preferences are more INTENSE 
than others.    Range voters can express that.  Borda voters cannot.

Surveys show that a lot of voters want to just vote for one candidate, plurality-style.
In range voting they can do that by voting (99, 0, 0, 0, 0, 0).
With Borda, they can't do it.


5.STRATEGY and 2-PARTY DOMINATION:
Most will, in an election like Bush v Gore v Nader 2000,
exaggerate their good and bad opinions of Bush and Gore by artificially ranking
them first and last, even if they truly feel Nader is best or worst.
They will do this in order to give their vote the "maximum possible impact"
so it is not "wasted".  Once they make this decision, with Borda,  Nader automatically
has to go in the middle slot, they have no choice about him.  If all
voters behave this way, then automatically the winner will be either
Bush or Gore.  Nader can NEVER win a Borda election with this kind of strategic voters.
(Unless it is an exact 3-way tie and the tie-break goes Nader's way,
which'll never happen in reality.)

In view of this, third parties would be silly to push Borda.  They should advocate range.

Analogously, in range voting, if the voters exaggerate and give Gore=99 and Bush=0 
(or the reverse) in order to get maximum impact and not waste their vote, then 
they are still free to give Nader 99 or 0 or anything in between.
Consequently, it would still be ENTIRELY POSSIBLE for Nader to win with range,
and without need of any kind of tie.

Think this kind of strategic thinking won't matter much?  Wrong.
The "National Election Study" showed that in 2000, among voters who honestly liked
Nader better than every other candidate, fewer than 1 in 10 actually voted
for Nader.  That was because of precisely this sort of strategic ploy - these
voters did not wish to "waste their vote" and wanted "maximum impact"
so they pretended either Bush or Gore was their favorite. (Same thing happened
with voters whose true favorite was Buchanan.)  In short, strategy
has an enormous impact in the real world, and over 90% of real
voters act strategically and not honestly, given the chance.
That is exactly why third parties always die out and the USA is
stuck with 2-party domination.

6.CLONES.
Suppose several near-identical "clone" candidates run.
Under plurality voting: they split the vote and all lose.
Under Borda voting: with enough clones, one is assured victory!
Range voting: clones don't matter, election result not affected.  Can't manipulate
the election by creating or abolishing clones.

Which would you prefer?

7.BAYESIAN REGRET (FOR STATISTICS NERDS):
Extensive computer simulations of millions of artificial "elections"
by W.D.Smith show that range voting is the best single-winner voting
system, among a large number compared by him (including Borda, Borda, Plurality, IRV
Condorcet, Eigenvector, etc.) in terms of a statistical yardstick called "Bayesian regret".
This is true regardless of whether the voters act honestly or strategically, whether
the number of candidates is 3,4, or 5, whether the number of voters is 5 or 200,
whether various levels of "voter ignorance" are introduced, and finally
regardless of which of several randomized "utility generators" are
used to generate election scenarios.

Smith's papers on voting systems are available at
   http://math.temple.edu/~wds/homepage/works.html
as #56, 59, 76, 77, 78, 79, 80.

8.SUMMARY:
Isn't the purpose of voting to provide information about your opinions?
Why would you want to have a system (Borda) that forces you to express less information,
when you can have one (Range) that permits you to express more?

If you think 2-party domination is a bad thing and would like to see a greater
diversity of parties and more voter choice,  then why would you want Borda (in which, with 
strategically-exaggerating voters, 3rd parties have no chance) when you could have Range?

And why would you want a system (Borda) that we know reacts very badly to strategic voters,
often giving a below-average candidate the victory?

Duh.  You want range voting.  Forget Borda voting.

-----end-----
