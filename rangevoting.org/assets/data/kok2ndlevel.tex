HOW TO DO 2-DIGIT RANGE VOTING ON ORDINARY PLURALITY VOTING MACHINES
-----------------------Warren D. Smith----Aug 2005------------------

We explain how to
extend Jan Kok's method for doing 1-digit range voting (range 0-9)
on any voting machine capable of handling multiple plurality contests
(i.e. every USA voting machine) to handle 2-digit range voting (range 0-99), 
if anybody should desire that.

The idea is to transform an election  Amy v Bob v Cal  to
6 plurality "elections" namely
(1)   Amy0s v Amy1s v Amy2s v ... v Amy9s
(2)   Amy0u v Amy1u v Amy2u v ... v Amy9u

(3)   Bob0s v Bob1s v Bob2s v ... v Bob9s
(4)   Bob0u v Bob1u v Bob2u v ... v Bob9u

(5)   Cal0s v Cal1s v Cal2s v ... v Cal9s
(6)   Cal0u v Cal1u v Cal2u v ... v Cal9u

where the "s" elections handle the most Significant digit
whereas the "u" elections handle the Units-place digit.

For example, a range voter wishing to give Amy a score of "57",
Bob a score of "32" and leave Cal's score blank,
would vote for:
   Amy5s and Amy7u,
   Bob3s and Bob2u,
and not vote in elections 5 and 6.

Equivalent picture of the 6 pseudo-elections:

(1)   Amy00 v Amy10 v Amy20 v ... v Amy90
(2)   Amy_0 v Amy_1 v Amy_2 v ... v Amy_9 

(3)   Bob00 v Bob10 v Bob20 v ... v Bob90
(4)   Bob_0 v Bob_1 v Bob_2 v ... v Bob_9 

(5)   Cal00 v Cal10 v Cal20 v ... v Cal90
(6)   Cal_0 v Cal_1 v Cal_2 v ... v Cal_9 

The people in charge of the voting machines would then
reconstruct Amy's summed score via
   10*Amy1s + 20*Amy2s + ... + 90*Amy9s + Amy1u + 2*Amy2u + ... + 9*Amy9u
(and Bob's and Cal's summed scores similarly)
and then divide Amy's summed score by the number of votes cast in election 1,
to get Amy's averaged score.
(And for Bob and Cal we would divide by #votes in elections 3 and 5, respectively.)

If you wanted to refuse to do these final divisions (i.e. base the range election on
totals rather than averages, i.e. equivalently regard "blank" votes as "zero") that would
of course be slightly easier.

WARNING:
A problem arises here which did not arise in the original Kok scheme for single-digit range voting
(allowed scores 0,1,2,3,4,5,6,7,8,9, and X for intentional blank).  That is
that voters here could vote, say, Amy2s BUT NOT vote in the Amyu election (2).
This would be a "partially blank" vote 2X where the unit-digit X was left unspecified by the 
voter.  (Also votes like X2 would be possible.)  Dumb plurality 
machines would be unable to handle these kinds of votes except if we agree to treat X as 0.
Then they would work, but we would thus be sacrificing the idea of having blank-votes
not affect a candidate's total -- they would now be treated as 0s and the total would
now be a sum, not an average of the non-blank scores.

THEREFORE WE RECOMMEND using single-digit range voting (0-9 & X) whenever old-style plurality 
machines are employed, but with gradual upgrading to 0-99 & X two-digit range voting when 
machines specifically designed for range voting instead of plurality voting, come along later.  
The old-style machines for 0-9 & X will need to have their results multiplied by 11 to
coexist with the newer 0-99 finer-grained machines.

DISCUSSION:
Although this may seem a bit complicated, the complexity from the point of view of the
VOTER is not bad: you can set this up on most kinds of voting machines (lever, punch
card, optical scan) so that it is pretty intuitive and straightforward for the 
voter to cast his range vote.  From the point of view of the election officials
it is more complicated, but still this is just maybe 30 extra entries into
a calculator per machine per election, which is a trivial increment to the large amount
of work they have to do anyway to run an election (for example pre-configuring all their
machines in the first place) so I am not much bothered by that.

(Side note:  It would be rather simpler and more advantageous if everybody had only 2 fingers, 
i.e. we all thought in binary rather than decimal...  which was my original plan
before Jan Kok came along, and which left me unsatisfied for obvious reasons...)

Naturally I would prefer it if special machines were developed designed specifically
for range voting, but the point of schemes like this one is they'd still permit
range elections to be held using low-tech plurality-election equipment if necessary.
