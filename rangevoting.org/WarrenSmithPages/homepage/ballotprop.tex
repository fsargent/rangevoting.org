POSSIBLE LANGUAGE FOR A BALLOT PROPOSITION FOR "DEMOCRACY IMPROVEMENT"
------------------Warren D. Smith-----------Nov 2004------------------

1. We hereby outlaw gerrymandering:  Each election district shall be required to 
  (a) have at most 6% population difference from the average among all districts;
  (b) have a boundary which always turns left or goes straight (never bends right),
        except at state boundaries or on unfordable rivers (wherever those bend right
        at places where the district boundary follows the state boundary or river-centerline)
  (c) their squared perimeter shall not exceed 20 times their area.
  (d) districts may be redrawn at most once between US censuses.

2. We hereby improve the voting system to allow voters to express more information
in their votes and to reconcile the desire to express their opinions honestly
in their vote, with the desire not to be strategically foolish:
  (a) each vote in an N-candidate election shall consist of one numerical score from 0 to 99
      awarded to each candidate (for example 57,0,34,99 could be one vote in a 4-candidate 
      election);
  (b) voters are allowed to leave entries blank if they desire not to express an
      opinion about that candidate (for example 57,0,*,99 where * denotes blank);
  (c) a candidate's "total" is the average of all his non-blank scores;
  (d) the candidate with the highest total wins.

In this "range voting" system, please note that awarding a high or low score to
candidate C in no way affects the battle purely between candidates A and B, so that
voters may feel free to express their honest opinion of C without fear of
hurting A.  Also please note that if candidate A has several "clones" A2, A3, 
then that will neither hurt nor help them, unlike in the old system where the 
clones might "split the vote" and lose.  Finally note that voters now can express
information about ALL the candidates instead of just one, or, by voting like (0,99,0,0),
can choose to act just like voters in the old system.

----------------------------------

The following US states allow statutes to be passed by "initiative":
AZ, AR, CA, CO, ID, MO, MT, NE, ND, OK, OR, SD, UT, WA.
The following US states allow constitutional amendments to be passed by
"initiative":
AZ, AR, CA, CO, FL, IL, MI, MO, MT, NE, NV, ND, OH, OK, OR, SD.

The usual procedure is, you formulate the wording very carefully of your
initiative, then you collect signatures to get it put on the ballot.
Your state's "secretary of state" office should probably have info on how to go 
about it.

-----end---------

