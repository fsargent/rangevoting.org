WHY RANGE VOTING IS BETTER THAN APPROVAL VOTING
--------------Warren D. Smith----Nov 2004------

1.WHAT THEY ARE.
Range voting: In an N-candidate election, each vote 
is an N-tuple of numbers each in the range 0 to 99.
The Kth number in the tuple is a "score" for candidate K.
You take the average of all the Kth entries to find
the average score for candidate K.  The candidate
with the highest average score wins. 
(Voters are allowed to leave an entry BLANK to denote
"don't know anything about that candidate."  Blank entries
not incorporated into average.)

Approval voting:  Same thing as range voting, except
each number, instead of being anywhere from 0 to 99 has to be 
either 0 ("disapprove") or 1 ("approve").

2.SIMPLICITY:
Approval voting wins this one: it obviously is simpler.
(But not by as much as you might think, since Jan Kok showed a simple way to
use any voting machine capable of handling multiple plurality contests - i.e.
all voting machines in the USA - to perform "single digit" range elections.
On lever machines, optical scan machines, and many kinds of punch card machines,
the voter finds it extremely easy and straightforward to use any of
Plurality, Approval, or single-digit Range Voting, with no real advantage in
simplicity for either.)

3.EXPRESSIVITY AND HONESTY.
Obviously, with range voting, you are free to cast an "approval style" vote
by only using 99s and 0s, never using anything in between - if you so desire.  
(And in fact, "strategic" voters will do exactly this.)
So range voters clearly can be either equally expressive, or
more expressive, than approval voters.

The more information voters can submit in their vote, the better we expect
the results of the election (from their point of view) to be.

So the question is: is approval voting better because it is simpler, or is range 
voting better because it allows more expressive freedom?

The answer hinges on how many voters will choose to be "honest" and how many will be
"strategic" (and how many will adopt some compromise).  That is because if
all voters are strategic, then range voting will simply degenerate to approval voting
and then there would be no reason to adopt RV, since it is more complicated.

However, Doug Greene and W.D.Smith did a range voting election among random 
voters emerging from polls in the 2004 Presidential election in New York State.
We found that (with 90% confidence) less than 25% of range voters vote
approval-style, and, further, that less than 43% even use the full range
(i.e. include both a 99 and a 0 in their vote).   We told them to vote as they
actually would if the real election were being held using range voting, and I
know I told my voters things like "the only meaning of the numbers you supply are
the numbers themselves; your job is to choose them to make it the most likely
the election will turn out in a way you like."  Now while I concede that
the 25 and 43 would undoubtably rise if our voters were exposed to more
information and thought longer than 1 minute (most or all of them had never heard of
range voting before they met us), it seems clear there will always be a substantial
fraction of range voters who do NOT want to vote approval-style.

This was surprising to me.  There is evidently a large psychological urge to be honest
and that urge seems to affect range voters to a far greater degree than plurality voters.
(It is known from National Election Study data that fewer than 1 in 10 of
2000 voters whose favorite was Nader, actually voted for Nader; same thing happened
with Buchanan - i.e. plurality voters are VERY strategic/dishonest.  Interesting
contrast, eh?  Range voting somehow stimulates people's honesty more than plurality does.)

Now.  Given that we know this fact about human beings,
it seems clear that range voting is better than approval voting, because
it allows more humans to satisfy their urges for honesty, and therefore
will produce election results that more honestly reflect voters' true opinions.

This can be a big effect favoring range over approval.

4.EXAMPLE: CALIFORNIA 2003 AND OTHER ELECTIONS WITH MANY CANDIDATES.

Consider the 135-candidate California Governor Recall election of 2003.
(The candidates were listed in random order!)
Suppose that there were 67 good candidates and 68 bad ones,
and suppose (optimistically) that almost all agreed about that goodness and badness.
(More realistically, I am afraid that many morons would vote for the muscular steroid movie star
and ignore the issues and the candidates' backgrounds, educations, and resumes like usual.
Obviously, I am not claiming the following scenario is entirely realistic, only that it
illustrates a way Range can yield superior results to Approval.)

If the election were run using approval voting, then everybody would approve
of the 67 good ones and not of the 68 bad ones.  Result: a 67-way tie.
In our scenario this tie would be broken fairly randomly by the few voters
who have bizarre biases and do not regard the good 67 as all good, e.g.
members of the candidates' immediate families.  (In the more pessimistic
movie-star scenario the movie-star would come out ahead of the other 66.)

But in fact suppose there is a gradation in quality among the top 67, which while
perhaps not clear to any one voter, collectively is clear.
With approval voting, there would be no way to express that and we would get
a random winner (or the movie star, who is probably worse-than-random - definitely Arnold
was worse than random candidates as far as his education - high school only -
and political background - zero - and ability to do his job without 15 women
coworkers going to major newspaper to complain he groped them - are concerned).

Meanwhile with range voting, even with a fairly small admixture of
honest voters mixed into a vast sea of strategic approval-style ones,
the 67-way tie would be broken honestly and the best quality among the top 67 would win.
If there were enough honest voters and enough of a gradation, the best
among the top 67 might even be able to win despite a decent fraction of
movie-star voting morons (depending on the relative numbers of honest voters and moronic
movie fans).  That is plausible because voters for Arnold were not actually all
morons.  There was a method to their madness.  Namely, they had a "coordination problem."
If all the republican voters voted for their favorite republican, then
the republican vote would be split and a democrat might win.  Horrors.   So the
republicans wanted to agree on *somebody*, doesn't matter much who,
as long as they could avoid such a vote split - and the most
logical one to agree on was Arnold Schwarzennegger because he already had a lead
(due to the movie-morons) and everybody had already heard of him.  
All other considerations such as his lack of education beyond high school fell by the wayside.
If Californians had instead been using a different-than-plurality voting
system in which vote-splitting did not exist (such as approval or range)
then only the ACTUAL morons would vote for Arnold; the rest, who were
only doing it to solve the plurality coordination problem, would now vote more
honestly, e.g. by approving all the republicans.  Arnold's vote
under approval or range would thus be far smaller than it was under plurality,
perhaps allowing the honest voters to win out over the morons.

To summarize:
So in this case we would expect range voting to elect the best candidate and approval voting
to elect somewhere around the 33rd-best candidate (in the no-morons scenario).
Meanwhile in the moron-heavy scenario, range would either yield the same result (Arnold)
as approval, or a better one.

This kind of scenario (many candidates) illustrates the great quality advantage that
range voting with honest voters can have over approval voting with honest voters;
and shows that vast quality improvement can also happen even with a mixture 
of, say, 90% strategic voters and 10% honest ones.

5.CALIFORNIA 2003 CONTINUED.
Suppose most voters knew something about, say, 5 of the 135 candidates, and 
nothing about the other 130.  With approval voting, they would probably "play it safe"
by disapproving of all the unknowns, resulting in a huge bias against lesser-known
candidates, as opposed to what we want, i.e. a bias against lesser-QUALITY candidates.
(That behavior was common in the sample of people I polled.
In the version of range and approval I have advocated above
they could leave those candidates BLANK.  But it is an experimental fact that
many voters prefer 0s to blanks in such cases.)

With range voting, they could give the unknowns all X, where X is a number they choose
to fill in everything else with.
For many voters quite possibly X would not be zero.  This would lessen this bias.

In riposte it was pointed out to me that Approval voters could simulate (a noisier
version of) range voting by tossing appropriately biased dice to select whether 
to approve each candidate.  However - be realistic.
That is not going to happen in the real world.  

So: range voting handles ignorant voters better than approval voting does - the latter
tends to introduce more artificial biases in that situation.

6.BAYESIAN REGRET (FOR STATISTICS NERDS):
Approval voting does quite badly (measured by Bayesian Regret)
compared to many other more expressive voting systems such
as Borda and Black, when there are 4 or more candidates and we have honest voters.
The more candidates, the worse approval does, with a very big gap developing.
This was shown in computer experiments by Merrill, later confirmed by me.

Approval is better than these systems - for strategic voters - but is worse for honest ones.

The joy of range voting is it is better BOTH for honest AND for strategic voters.
Computer Bayesian regret measurements by me show range voting as good or better
than every other voting system tried including Approval, Plurality, IRV, Condorcet, Borda,
in a vast number of different kinds of scenarios with ither honest or
strategic voters, whether there are 5 voters or 200, 3,4, or 5 candidates,
different kinds of "utility generators", and different levels of "voter ignorance".
For my papers on voting see 
   http://math.temple.edu/~wds/homepage/works.html    #56, 59, 76-80.

7.POTENTIAL FOR TIES AND NEAR-TIES:
Remember how Bush v Gore, Florida 2000, was officially decided by only 537 votes,
and this caused a huge crisis?  Ties and near-ties are bad.  

Exact ties are much less likely in range voting than approval voting, simply because
there is a large range 0-99 rather than 0-1.  (Near-ties are probably also less likely,
but not tremendously so.)

8.UNFAIR ADVANTAGE FOR SMARTER VOTERS?
It was suggested to me that an "advantage" of approval over range
was that in range voting, some voters will be honest and others will be
strategic (i.e vote approval-style), and the latter will have an "unfair" advantage.
Meanwhile with approval voting, all voters will (to a good approximation) be
forced to act strategic, like it or not.  So this "unfairness" would vanish.

To this I reply:  if you buy this logic, then it would follow that you also support forcing
voters to vote Democrat or Republican in the present plurality system.  Since,
after all, voting third-party is (while honest) strategically stupid.  Those
who like third-party candidates and who are stupid enough to honestly say so
are placed at an "unfair" disadvantage and we should force them to vote democratic
or republican to protect them from themselves.  Right?

Well, no.  Nobody supports that.  This is not an "unfair" advantage anyhow, since
nobody is forcing any voter to do anything and all voters are treated the same.

9.SUMMARY:
Isn't the purpose of voting to provide information about your opinions?
Why would you want to have a system (Approval) that forces you to express less information,
when you can have one (Range) that permits you to express more?

Range is better than approval for honest voters and the same as approval for
strategic voters, and in practice there are a lot of both kinds of voter.
Consequently there is a big quality advantage for range voting, especially
if there are a lot of candidates, and this advantage is well worth the
extra complexity of range.

-----end-----
