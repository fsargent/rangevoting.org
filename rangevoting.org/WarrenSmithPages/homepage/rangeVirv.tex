WHY RANGE VOTING IS BETTER THAN IRV (Instant Runoff Voting)
--------------Warren D. Smith----Nov 2004------------------


1.WHAT THEY ARE.
Range voting: In an N-candidate election, each vote
is an N-tuple of numbers each in the range 0 to 99.
The Kth number in the tuple is a "score" for candidate K.
You take the average of all the Kth entries to find
the average score for candidate K.  The candidate
with the highest average score wins.
(Voters are allowed to leave an entry BLANK to denote
"don't know anything about that candidate."  Blank entries
not incorporated into average.)

IRV: Each vote is an ordering of the N candidates
from best to worst.  (Voters are NOT allowed to omit a
candidate they know nothing about, and are NOT allowed
to regard two candidates as EQUAL.)
We proceed in "rounds".  Each round, the candidate top-ranked
by the fewest voters is eliminated.  (Ties must be broken by some
mans, such as a coin flip.)  After N-1 rounds, only
one remains - the winner.

2.EXAMPLES:
Say the N=3 candidates are named Alice, Bob, and Carl,
and the 9 range votes for them are:
           A   B   C
---------------------
voter1:   99  77   0
voter2:   99  98   *
voter3:   87   0  71       RANGE EXAMPLE.
voter4:   99   1  98
voter5:   51  99   *
voter6:   26  98   5
voter7:   10  96   *
voter8:    0  70  99
voter9:    0   1  99
---------------------
total    441 540 372
average   49  60  62
(The *'s denote "intentionally left blank" by that voter
to denote the fact he wishes to express ignorance about Carl and
wants to leave the decision about his score to other, more knowledgeable
voters)... and C wins.

The corresponding 9 IRV votes (assuming all the voters ignorant
about C "play it safe" by ranking him last) are
voter1:   A>B>C
voter2:   A>B>C
voter3:   A>C>B
voter4:   A>C>B       IRV EXAMPLE.
voter5:   B>A>C
voter6:   B>A>C
voter7:   B>C>A
voter8:   C>B>A
voter9:   C>B>A
Then C is eliminated in the first round, at which point the second and
final round is won by B by 5-to-4 over A.

3.MONOTONICITY and HONESTY:
In range voting, if any set of voters increase a candidate's score,
it obviously can help him, but cannot hurt him.  That is called monotonicity.

One of IRV's flaws is that it is not monotonic and dishonesty can pay.
In the example, suppose voter1, instead of honestly stating
her top-preference was A, were to dishonestly vote C>A>B,
i.e. pretending great LOVE for her truly hugely-hated candidate C,
and pretending a LACK of affection for her true favorite A.
In that case the first round would eliminate either C or B
(suppose a coin flip says B) at which point A would win the
second round 5-to-4 over C.  (Meanwhile if C still were eliminated
by the coin flip then B would still win over A in the final
round as before.)
In other words: in 3-candidate IRV elections, lying can help. 
Indeed, lying in bizarre ways can help.

In this IRV example, voter1's vote for her true favorite (here A)
actually CAUSED him to LOSE!!  Analysis by W.D.Smith shows that 
slightly more than 1% of 3-candidate IRV elections are non-monotonic.
(Whenever this happens we can expect tremendous rage from the "robbed
winner" and calls for reform of the idiotic IRV voting system, and for the
heads of the idiotic activists who originally advocated IRV and got us into this mess.)

In contrast, in 3-candidate range elections, it is never a smart move
for any voter to be dishonest about her relative ordering of A, B, and C;
if an intelligent range voter feels that A>B, she will never provide
a vote in which B>A.

4.SIMPLICITY:
Range voting is simpler than IRV.
If you don't believe me, try writing a computer program to do both.
The range voting program will be shorter.
Also, range voting is simpler in the sense that it requires fewer operations
to perform an election.  In a V-voter, N-candidate election,
range voting takes roughly 2*V*N operations.  However, IRV voting
takes roughly that many operations every 2 rounds.  In a 135-candidate
election like California Gubernatorial 2003, IRV would require about 67 times
as many operations.  (In fact, range voting is simple enough that it could
be done with hand calculators, if necessary.)

5.STRATEGY and 2-PARTY DOMINATION:
A lot of voters will, in an election like Bush v Gore v Nader 2000,
exaggerate their good and bad opinions of Bush and Gore by artificially ranking
them first and last, even if they truly feel Nader is best or worst.
They will do this in order to give their vote the "maximum possible impact"
so it is not "wasted".  Once they make this decision, in IRV, Nader automatically
has to go in the middle slot, they have no choice about him.  If all
voters behave this way, then automatically the winner will be either
Bush or Gore.  Nader can NEVER win an IRV election with this kind of strategic voters.
(Unless it is an exact 3-way tie and the tie-break goes Nader's way,
which'll never happen in reality.)

In view of this, third parties are silly to push IRV.  They should advocate range.
(Fact: The countries that use IRV, namely Ireland and Australia, are 2-party dominated.)

Analogously, in range voting, if the voters exaggerate and give Gore=99 and Bush=0 
(or the reverse) in order to get maximum impact and not waste their vote, then 
they are still free to give Nader 99 or 0 or anything in between.
Consequently, it would still be ENTIRELY POSSIBLE for Nader to win with range,
and without need of any kind of tie.

Think this kind of strategic thinking won't matter much?  Wrong.
The "National Election Study" showed that in 2000, among voters who honestly liked
Nader better than every other candidate, fewer than 1 in 10 actually voted
for Nader.  That was because of precisely this sort of strategic ploy - these
voters did not wish to "waste their vote" and wanted "maximum impact"
so they pretended either Bush or Gore was their favorite. (Same thing happened
with voters whose true favorite was Buchanan.)  In short, strategy
has an enormous impact in the real world, and over 90% of
voters act strategically and not honestly, given the chance.
That is exactly why third parties always die out and we are stuck with 2-party domination.

6.POTENTIAL FOR TIES AND NEAR-TIES:
Remember how Bush v Gore, Florida 2000, was officially decided by only 537 votes,
and this caused a huge crisis?  Ties and near-ties are bad.  In IRV
there is potential for a tie or near-tie every single round.  That makes
the crisis-potential inherent in IRV much larger than it has to be.  That also
means that in IRV, every time there is a near-tie among two no-hope
candidates, we have to wait, and wait, and wait, until we have the EXACT vote
totals for the Flat-Earth candidate and for the Alien-Kidnapping candidate 
since every last absentee ballot has finally arrived...
before we can finally decide which one to eliminate in the first round.
Only then can we proceed to the second round.  We may not find out the winner
for a long time.  The precise order in which the no-hopers are eliminated matters
because it will affect the results of future rounds.

Don't think this will happen?  In the CA gubernatorial recall election of 2003,
  D. (Logan Darrow) Clements   got  274  votes, beating   Robert A. Dole's  273.
Then later on in the same election,
  Scott W. Davis               got  382 votes, beating    Daniel W. Richards's 381.
Then later on in the same election,
  Paul W. Vann        got 452 and   Michael Cheli 451 votes.
Then later on in the same election,
   Kelly P. Kimball  got  582  and  Mike McNeilly  581 votes.
Then later on in the same election,
  Christopher Ranken got  822  and  Sharon Rushford 821 votes.
Have you had enough yet?    Eventually Schwarzenegger won.  
Oh, was that what you wanted to know?

(Incidentally, imagine the horror if each voter were required to provide a preference ranking
of all 135 candidates in this race in the first place.  Meanwhile with range voting they
just rank the ones they know about and leave the rest blank, or they could
opt to "fill in all the rest with X" where X is a number they specify.  Much
less labor for the voter.)

Think the CA 2003 recall was a monster not likely to recur?  Actually in New South Wales
(Australia) in 1974 they had an IRV election with 73 candidates, and voters were
required by law to (1) vote and (2) rank all 73 of them (none missed) - but they also
were given the option of voting a pre-prepared straight-party ticket instead.

Meanwhile, in range voting, the only thing that matters is the top scorer.
Ties for 5th place, do not matter in the sense they do not lead to crises.  
Furthermore, because all votes are real numbers
0-99 rather than discrete and from a small set, EXACT ties are even less likely still.
(And if the range were, say, 0-999 then they would be even less likely still.)
Exact ties in range elections can thus be rendered extremely unlikely.

7.COMMUNICATION NEEDS and NIGHTMARE POTENTIAL:
Suppose a 1,000,000-voter N-candidate election is carried out at 1000 different 
polling locations, each with 1000 voters.  In range voting, each location can then 
compute its own subtotal N-tuple and send it to the central agency, which then adds up 
the subtotals and announces the winner.  That is very simple.  That is a very small
amount of communication (1000*N numbers), and all of it is one-way.  Furthermore, if
some location finds it made a mistake or forgot some votes,
it can send a corrected subtotal, and the central agency can then easily 
correct the full total by doing FAR less work than making everybody
completely redo everything.  

But in IRV voting, we cannot do these things because IRV is not additive.
There is no such thing as a "subtotal" in IRV.    In IRV every single vote
may have to be sent individually to the central agency (1,000,000*N numbers, 
i.e. 1000 times more communication).  [Actually there are clever ways to reduce this, 
but it is still bad.] If the central agency then
computes the winner, and then some location sends a correction, that may require
redoing almost the WHOLE computation over again.  There could easily be 100 such
corrections and so you'd have to redo everything 100 times.  Combine this
scenario with a near-tie and legal and extra-legal battle like in Bush-Gore
Florida 2000 over the validity of every vote, and this adds up to a complete
nightmare for the election administrators.

Don't think IRV nightmares really can happen?  Here's what happened 
when San Fransisco adopted IRV:

  Preferential voting software breaks down in San Francisco
  Thu, 4 Nov 2004 10:07:12 PST

  In the election of 2 Nov 2004, San Francisco's district supervisor
  election used ranked-choice voting for the first time.
  It went just fine on Tuesday during the election.
  Preliminary results showed candidates in three districts had won by a clear
  majority (so no reranking-rounds were needed), whereas the other four seats 
  remained to be determined by the preferential ballot counting process.
  The computer processing broke down completely on Wednesday afternoon when election
  workers began to merge the first, second, and third choices into the
  program that is supposed to sequentially eliminate low-vote candidates
  and redistribute voters' second and third choices accordingly. 

  However [fortunately], no San Francisco ballots were lost, because each ballot has a
  paper trail.

  The software is provided by ES&S (Election Systems and Software, in
  Omaha). This system has undergone federal and state testing, as well
  as pre-election testing in which everything seemed to work perfectly.
  The results of the four contested supervisors' races are expected to be delayed 
  up to two weeks. 
  [Suzanne Herel, *San Francisco Chronicle*, 4 Nov 2004, front page continued on A7]

8. VOTER EXPRESSIVITY:
In range voting, voters can express the idea that they think 2 candidates are equal.
In IRV, they cannot.  (There are modifications of IRV which permit equalities, but
they are much more complicated.  They involve considering "every possible compatible
ordering."   In fact they are so complicated I doubt most voters will ever be able
to fully describe how they work.)

Range voters can express the idea they are IGNORANT about a candidate and want to leave the
task of rating him to other, hopefully more knowledgeable voters.  In IRV, they
can't choose to do that.

IRV voters who decide, in a 3-candidate election, to rank A top and B bottom,
then have no choice about C - they have to middle-rank him and can in no way express
their opinion of C.   In range voting, they can.

If you think  A>B>C>D>E,  undoubtably some of your preferences are more INTENSE 
than others.    Range voters can express that.  IRV voters cannot.

A lot of voters want to just vote for one candidate, plurality-style.
In range voting they can do that by voting (99,0,0,0,0,0).
In IRV, they can't do it.  (Actually in some variants of IRV, they can.
It depends.  In many Australian elections, full orderings of all candidates
are required or your vote is invalid.  But in Ireland and Malta, just naming
your first few choices and not the others is allowed.)

9.BAYESIAN REGRET (FOR STATISTICS NERDS):
Extensive computer simulations of millions of artificial "elections"
by W.D.Smith show that range voting is the best single-winner voting
system, among a large number compared by him (including IRV, Borda, Plurality,
Condorcet, Eigenvector, etc.) in terms of a statistical yardstick called "Bayesian regret".
This is true regardless of whether the voters act honestly or strategically, whether
the number of candidates is 3,4, or 5, whether the number of voters is 5 or 200,
whether various levels of "voter ignorance" are introduced, and finally
regardless of which of several randomized "utility generators" are
used to generate election scenarios.

Smith's papers on voting systems are available at
   http://math.temple.edu/~wds/homepage/works.html
as #56, 59, 76, 77, 78, 79, 80.

10.USA HISTORY LESSON
About 2 dozen US cities have over the years adopted IRV, the largest being New York City in 1936.
However, almost all of those cities later decided to get rid of it.  (One of the exceptions
is Cambridge MA, which still uses it.)  I guess IRV advocates should be asking themselves -
why did these cities go back?  But oddly enough, they never seem to ask themselves that.  In 
fact, they usually don't even know it.

11. OCCASIONAL EXTREMELY NASTY AND ILLOGICAL RESULTS.
#voters    their vote   simplified 
-------    ----------   ----------
50         A>B>C>D>E       A>B
51         B>A>C>D>E        B
100        C>D>B>E>A       C>D
53         D>E>C>B>A        D
49         E>D>C>B>A       E>D

In this example from  election-methods.org,
the centrist candidate C is the favorite of far more voters than anybody else,
and not only would win a head-to-head contest with any single opponent
(based on these votes) but in fact would do so 
by approximately a 2:1 MARGIN.     So the "right"  winner here clearly seems to be C.
and almost every ranked-ballot voting system would indeed elect C.
But not IRV.  IRV elects D!
(IRV only examines the preference relations in the "simplified" votes, and
ignores the others.)

You can make this example even sicker: [the reader may enjoy this easy exercise]
construct a scenario in which the Condorcet Winner beats each other candidate pairwise by a 
99:1 margin (!), but nevertheless is eliminated in the very first round of IRV voting!

Meanwhile: range voting never seems to do anything especially illogical and
hard to justify.

12.SUMMARY:
Isn't the purpose of voting to provide information about your opinions?
Why would you want to have a system (IRV) that forces you to express less information,
when you can have one (Range) that permits you to express more?

Why would you want a more complicated system, with more nightmare potential, more
tie-potential, longer delays, more chance of extremely goofy illogicality,
and vastly larger communication needs (IRV) when you 
can avoid all that with Range?

Why would you want a system where voting for your favorite can actually hurt him (IRV) when
you could just have a monotonic system (Range)?

If you think 2-party domination is a bad thing and would like to see a greater
diversity of parties and more voter choice,  then why would you want IRV (in which, with 
strategically-exaggerating voters, 3rd parties have no chance, and
which in Australia and Ireland still led to 2-party domination) when you could have Range?

And why wouldn't you want the BEST system, as measured by "Bayesian regret" (Range)??

Duh.  You want range voting.  Forget IRV voting.

-----end-----
