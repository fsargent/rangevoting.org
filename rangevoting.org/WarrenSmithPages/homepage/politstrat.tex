STRATEGY TO MAKE THE USA ADOPT RANGE VOTING
--------Warren D. Smith-------Nov 2004-----wds@math.temple.edu-----

FIRST, I would like to get all the top 3
minor parties (Greens, Libertarians, and 
Constitutionalists) to AGREE to make a unified push
for range voting.  

It is clear that no 3rd party has any chance under the 
present Plurality voting system.  So their top priority must
be to make the USA adopt a better voting system than plurality.
Survival is the highest priority.  You must do this thing or die.

Because range voting is the best single-winner voting system,
I would recommend unifying behind IT instead of some worse system.

The US Constitution merely says there should be votes; it does not
specify which system should be used.  Thus, switching
from plurality to range could be done by enacting state laws,
passing state ballot propositions, or enacting federal laws, without
need of a constitutional amendment (although I would ultimately recommend 
having one).  Indeed, some states have, in previous times, used
other voting systems, such as several Southern states once used
the "Bucklin system" and Illinois previously used the "cumulative vote".


SECOND, I believe the route most likely to work for me is to 
put propositions on the ballot in states (such as California) which
allow them, saying "switch to range voting by the year X+5."

However, just getting a range-proposition put on the ballot is not good enough.
I did a poll of random voters with Doug Greene.  In NY State we 
found that 70 voters  (once range voting was briefly explained to them and
they filled out a sample range ballot)  preferred to stay with the
present Plurality voting system, while 44 wanted to switch to Range Voting.  
So any such ballot proposition is gauranteed to lose if voters can only think about 
the issue for 1 minute.  The only way it can win is if a great deal of
media attention is devoted to voting systems, so that a lot of people think
about it for a good deal longer than 1 minute.  The only way that can happen is
either 
  (a) to acquire and use millions of dollars in advertising
      (perhaps a wealthy philanthropist like George Soros?) 
or
  (b) to get it on the ballot very early, way before the actual election,
    and then make it such an important issue that the media has to examine it 
    heavily, even without you doing any advertising.
also
  (c) I recommend wording it as a combined proposition both advocating range voting
    AND outlawing gerrymandering.  That will make it more beneficial and more likely
    to pass.
Probably all three will be necessary.  It would also be good to have a web site
www.rangevoting.org.


THIRD, once range voting has been adopted by a few states, the rest of them hopefully
will see that it works well, and the ball can get rolling for its further adoption
in more states.  Further, by that time I will hopefully have a book.


